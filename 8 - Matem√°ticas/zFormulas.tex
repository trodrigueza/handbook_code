\begin{center}
{\renewcommand{\arraystretch}{1.4}
\begin{tabular}{|p{0.28\columnwidth}|p{0.68\columnwidth}|}
\hline
\multicolumn{2}{|c|}{} \\
\multicolumn{2}{|c|}{\textbf{PERMUTACIÓN Y COMBINACIÓN}} \\
\multicolumn{2}{|c|}{} \\ \hline

Combinación (Coeficiente Binomial) &
Número de subconjuntos de $k$ elementos escogidos de un conjunto con $n$ elementos:
$\displaystyle \binom{n}{k} = \binom{n}{n-k} = \frac{n!}{k!(n-k)!}$ \\ \hline

Combinación con repetición &
Número de grupos formados por $n$ elementos, partiendo de $m$ tipos:
$\displaystyle CR_{m}^{n} = \binom{m+n-1}{n} = \frac{(m+n-1)!}{n!(m-1)!}$ \\ \hline

Permutación &
Número de formas de agrupar $n$ elementos (importa el orden, sin repetición):
$P_{n} = n!$ \\ \hline

Permutación múltiple &
Elegir $r$ elementos de $n$ posibles con repetición:
$\displaystyle n^{r}$ \\ \hline

Permutación con repetición &
$n$ elementos donde el primero se repite $a$ veces, el segundo $b$ veces, el tercero $c$ veces, \ldots:
$\displaystyle PR_{n}^{a,b,c\ldots} = \frac{P_{n}}{a!\,b!\,c!\ldots}$ \\ \hline

Permutaciones sin repetición &
Número de formas de agrupar $r$ elementos de $n$ disponibles, sin repetición:
$\displaystyle \frac{n!}{(n-r)!}$ \\ \hline

\multicolumn{2}{|c|}{} \\
\multicolumn{2}{|c|}{\textbf{DISTANCIAS}} \\
\multicolumn{2}{|c|}{} \\ \hline

Distancia Euclideana &
$\displaystyle d_{E}(P_{1},P_{2}) = \sqrt{(x_{2}-x_{1})^{2}+(y_{2}-y_{1})^{2}}$ \\ \hline

Distancia Manhattan &
$\displaystyle d_{M}(P_{1}, P_{2}) = |x_{2} - x_{1}| + |y_{2} - y_{1}|$ \\ \hline

\multicolumn{2}{|c|}{} \\
\multicolumn{2}{|c|}{\textbf{CIRCUNFERENCIA Y CÍRCULO}} \\
\multicolumn{2}{|c|}{} \\ \hline

\multicolumn{2}{|p{0.96\columnwidth}|}{Considerando $r$ como el radio, $\alpha$ como el ángulo del arco o sector, y $(R,r)$ como radio mayor y menor respectivamente.} \\ \hline

Área                   & $\displaystyle A = \pi r^{2}$ \\ \hline
Longitud               & $\displaystyle L = 2\pi r$  \\ \hline
Longitud de un arco    & $\displaystyle L = \frac{2\pi r\,\alpha}{360}$  \\ \hline
Área sector circular   & $\displaystyle A = \frac{\pi r^{2} \alpha}{360}$ \\ \hline
Área corona circular   & $\displaystyle A = \pi (R^{2} - r^{2})$ \\ \hline

\multicolumn{2}{|c|}{} \\
\multicolumn{2}{|c|}{\textbf{TRIÁNGULO}} \\
\multicolumn{2}{|c|}{} \\ \hline

\multicolumn{2}{|p{0.96\columnwidth}|}{Considerando $b$ como la longitud de la base, $h$ como la altura, letras minúsculas como la longitud de los lados, letras mayúsculas como los ángulos, y $r$ como el radio de circunferencias asociadas.} \\ \hline

Área conociendo base y altura & $\displaystyle A = \frac{1}{2}\, b\, h$ \\ \hline
Área conociendo 2 lados y el ángulo que forman & $\displaystyle A = \frac{1}{2}\, a\, b\, \sin C$ \\ \hline
Área conociendo los 3 lados & $\displaystyle A = \sqrt{p(p - a)(p - b)(p - c)}$ con $\displaystyle p = \frac{a + b + c}{2}$ \\ \hline
Área de un triángulo circunscrito a una circunferencia & $\displaystyle A = \frac{abc}{4r}$ \\ \hline
Área de un triángulo inscrito a una circunferencia & $\displaystyle A = r\!\left(\frac{a+b+c}{2}\right)$ \\ \hline
Área de un triángulo equilátero & $\displaystyle A = \frac{\sqrt{3}}{4}\, a^{2}$ \\ \hline

\multicolumn{2}{|c|}{} \\
\multicolumn{2}{|c|}{\textbf{RAZONES TRIGONOMÉTRICAS}} \\
\multicolumn{2}{|c|}{} \\ \hline

\multicolumn{2}{|p{0.96\columnwidth}|}{Considerando un triángulo rectángulo de lados $a,b,c$, con vértices $A,B,C$ (cada vértice opuesto al lado cuya letra minúscula coincide con él) y un ángulo $\alpha$ con centro en el vértice $A$. $a$ y $b$ son catetos, $c$ es la hipotenusa:} \\ \hline

\multicolumn{2}{|p{0.96\columnwidth}|}{$\displaystyle \sin\alpha = \frac{\text{cateto opuesto}}{\text{hipotenusa}} = \frac{a}{c}$} \\ \hline
\multicolumn{2}{|p{0.96\columnwidth}|}{$\displaystyle \cos\alpha = \frac{\text{cateto adyacente}}{\text{hipotenusa}} = \frac{b}{c}$} \\ \hline
\multicolumn{2}{|p{0.96\columnwidth}|}{$\displaystyle \tan\alpha = \frac{\text{cateto opuesto}}{\text{cateto adyacente}} = \frac{a}{b}$} \\ \hline
\multicolumn{2}{|p{0.96\columnwidth}|}{$\displaystyle \sec\alpha = \frac{1}{\cos\alpha} = \frac{c}{b}$} \\ \hline
\multicolumn{2}{|p{0.96\columnwidth}|}{$\displaystyle \csc\alpha = \frac{1}{\sin\alpha} = \frac{c}{a}$} \\ \hline
\multicolumn{2}{|p{0.96\columnwidth}|}{$\displaystyle \cot\alpha = \frac{1}{\tan\alpha} = \frac{b}{a}$} \\ \hline

\multicolumn{2}{|c|}{} \\
\multicolumn{2}{|c|}{\textbf{PROPIEDADES DEL MÓDULO (RESIDUO)}} \\
\multicolumn{2}{|c|}{} \\ \hline

Propiedad neutro & $(a \% b) \% b = a \% b$ \\ \hline
Propiedad asociativa en multiplicación & $(ab) \% c = ((a \% c)(b \% c)) \% c$ \\ \hline
Propiedad asociativa en suma & $(a + b) \% c = ((a \% c) + (b \% c)) \% c$ \\ \hline

\multicolumn{2}{|c|}{} \\
\multicolumn{2}{|c|}{\textbf{CONSTANTES}} \\
\multicolumn{2}{|c|}{} \\ \hline

Pi & $\displaystyle \pi = \arccos(-1) \approx 3.14159$ \\ \hline
e  & $\displaystyle e \approx 2.71828$ \\ \hline
Número áureo & $\displaystyle \phi = \frac{1 + \sqrt{5}}{2} \approx 1.61803$ \\ \hline

\end{tabular}
}
\end{center}
